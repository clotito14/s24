\documentclass{IEEEtran}

\hbadness=99999

% Packages
\usepackage{amsmath}
\usepackage{physics}
\usepackage[cmintegrals]{newtxmath}
\usepackage{graphicx}
\usepackage{xurl} % Makes urls better
\graphicspath{{./images}}

% Title Stuff
\title{Semiconductor Bandstructures and Engineering Related Parameters}
\author{Chase A. Lotito, \textit{SIUC Undergraduate}}
\date{}

% Makes a header!
\markboth{ECE447 --- Semiconductor Devices --- Project 2, March 2024}{Shell \MakeLowercase{\text
it{et al.}}: A Novel Tin Can Link}

\begin{document}

\maketitle % Makes the title

% ABSTRACT
\begin{abstract}
    % [A brief statement on what you plan to do in this project.]
    This experiment is oriented around the bandstructure of a semiconductor, and what factors can change it. Using the "Band Structure Lab" from nanoHUB.org, we calculate and plot the bandstructures for bulk Si, Ga, and GaAs, and nanowire Si. We observe the different energies for a carrier's momentum, the degeneracy in the valence and conduction bands, and how applying strain to a semiconductor can change all of these.
\end{abstract}

\section{Introduction}

The bandstructure of a semiconductor, which is derived from solutions to the Schr\"{o}dinger equation, shows the energies a charge carrier occupies for given a given momentum. From the bandstructure, you also find the bandgap energy (direct or indirect), effective mass, degeneracy, and other properties about the semiconductor.

\section{Design Method}
% [i.e. the simulator used and a reference to it.]
For this bandstructure experiment, the \textit{Band Structure Lab} from nanoHUB.org was used \cite{sim}.

This simulator allows calculations of the bandstructure for bulk materials, nanowires, and ultra-thin bodies. Within each simulation we can specify spin-orbit coupling, uniaxial strain (among other types), the regions in \(k\)-space we are interested in, and how fine we want the results.

For discussions regarding effective mass:

\begin{equation}\label{eq:curvature-effective-mass}
    \pdv[2]{E}{k} = \frac{\hbar^2}{m^*}
\end{equation}

\section{Discussion on Design and Analysis}
% [Describe design and analysis. attach necessary graphs, snapshots of simulator pages, and/or codes if used.]

\subsection{Bulk Materials}

\textbf{The First Set - No Spin-Orbit Coupling}

\subsubsection{Effective Mass and Direct/Indirect Bandgap}

We can see the bandstructure for silicon is Figure \ref{fig:si-first-set}. The yellow curve is the edge of the conduction band. Along this curve, it appears to have the tightest curvature at point W, which means the electron effective mass is smallest at point W. Silicon shows itself to be a indirect-bandgap. This is due to the conduction band minimum (CBM) occurring at a different point in \(k\)-space than the valence band maximum (VBM). For Si, the CBM occurs at \(k \approx 10 ~ nm^{-1}\), and the VBM occurs at \(\Gamma\).

Figure \ref{fig:ge-first-set} shows the Germanium bandstructure. The conduction-band curvature is the narrowest at \(\Gamma\), so the electron effective mass is the smallest at \(\Gamma\). Also, the CBM occurs at \(L\), and the VBM occurs at \(\Gamma\), therefore Ge is an indirect-bandgap.

Figure \ref{fig:gaas-first-set} shows the Gallium Arsenide bandstructure. The conduction-band curvature is the narrowest at \(\Gamma\), so the electron effective mass is the smallest at \(\Gamma\). Also, the CBM and VBM occur at \(\Gamma\), therefore Ge is a direct-bandgap. 

\subsubsection{Negative Electron Effective Mass}

Given Eq. \ref{eq:curvature-effective-mass}, negative values for electron effective mass occur when the curvature of the conduction band is negative (concave down). For Si, this occurs at \(\Gamma\) and W. For Ge and GaAs, this occurs before and after \(\Gamma\) and \(k \approx 16 ~ nm^{-1}\). Physically, for an electron to have a negative mass, at that particular direction in the crystal, the electron will appear to accelerate upwards--float.

% TODO: Comment on the shape (spherical or ellipsoidal) of the constant energy surface for the conduction band minima

\subsubsection{Shape of Constant Energy Surfaces}

To talk about the shapes of the constant energy surfaces for the conduction band minima, we need to see where the conduction band minima occur in the Brillouin Zone, and their curvature.

The Brillouin Zone, depicted in Figure \ref{fig:brillouin}, represents the semiconductor lattice in \(k\)-space (or reciprocal space) for a face-centered cubic crystal \cite{brillouin}.

For Silicon, the CBM occurs in its entirety near point \(X\), which occurs at 6 faces perpendicular to \(k_x\), \(k_y\), and \(k_z\). Since the bandstructure we calculated is 1D, if we imagine revolving the shallow parabola around the Si CBM to achieve a 3D shape, we would get 6 ellipsoids at the \(X\)-point in the Brillouin Zone. 

For Germanium, the CBM occurs at the \(L\)-point, which translates to 8 diagonally-facing faces in the Brillouin Zone. The curve around the CBM is also a shallow parabola, so when revolved into a 3D shape, we get 8 ellipsoids.

For Gallium Arsenide, the CBM occurs directly at the \(\Gamma\)-point, and its curvature is similar to a semicircle. When translated into 3D \(k-space\), the constant energy level would be spherical.

\begin{figure}[!ht] 
    \centering
    \includegraphics*[width = 6cm]{brillouinZone.png}
    \caption{Brillouin Zone}
    \label{fig:brillouin}
\end{figure}    

% Comment on the degeneracy of the valence bands

\subsubsection{Degeneracy in Valence Bands}

Observing closely, 3 bands overlap at the VBM for Ge, so Ge has three-fold degeneracy in the valence band. The same occurs for Si and GaAs. All of these semiconductors showing heavy-holes and light-holes at \(\Gamma\).

% Which material offers the smallest hole effective mass? 

\subsubsection{Smallest Hole Effective Mass}

Again from Figure \ref{fig:ge-first-set}, we can see that the red curve, which shows the light-holes in the valence band for Ge, has the narrowest curvature as compared to both Si and GaAs; therefore, Ge has the smallest hole effective mass.

% Graph of Si Bandstucture - First Set
\begin{figure}[!ht] 
    \centering
    \includegraphics*[width = 6cm]{si-bands-firstset.png}
    \caption{Silicon bandstructure, bulk, no spin-orbit coupling}
    \label{fig:si-first-set}
\end{figure}    

\begin{figure}[!ht] 
    \centering
    \includegraphics*[width = 6cm]{ge-bands-firstset.png}
    \caption{Germanium bandstructure, bulk, no spin-orbit coupling}
    \label{fig:ge-first-set}
\end{figure}    

\begin{figure}[!ht] 
    \centering
    \includegraphics*[width = 6cm]{gaas-bands-firstset.png}
    \caption{Gallium Arsenide bandstructure, bulk, no spin-orbit coupling}
    \label{fig:gaas-first-set}
\end{figure}    

\textbf{The Second Set - With Spin-Orbit Coupling}

When zooming in, the valence band structure at locations of degeneracy show small gaps in-between, this is most prominent at the VBM for all materials. From a distance it seems unnoticeable, the gaps are very small--without spin-orbit coupling the bands were touching, therefore degenerate. I suspect these small delineations in energy are caused by Pauli's Exclusion principle detecting similar carrier spins, and forcing their energies to split.

\begin{figure}[!ht] 
    \centering
    \includegraphics*[width = 6cm]{si-bands-secondset.png}
    \caption{Silicon bandstructure, bulk, with spin-orbit coupling}
    \label{fig:si-second-set}
\end{figure}    

\begin{figure}[!ht] 
    \centering
    \includegraphics*[width = 6cm]{ge-bands-secondset.png}
    \caption{Germanium bandstructure, bulk, with spin-orbit coupling}
    \label{fig:ge-second-set}
\end{figure}    

\begin{figure}[!ht] 
    \centering
    \includegraphics*[width = 6cm]{gaas-bands-secondset.png}
    \caption{Gallium Arsenide bandstructure, bulk, with spin-orbit coupling}
    \label{fig:gaas-second-set}
\end{figure}    

\textbf{The Third Set - With Uniaxial Strain}

% TODO: Show that the strain breaks the degeneracy in the valence band. What implications does this finding have on hole transport (current) in a device?

For the third set, we apply a strain of \(\epsilon = -0.05\), which represents a compressive strain, in the [100] direction. In Figure \ref{fig:si-third-set}, we see distinctly the once degenerate valence bands have now separated. This is important for hole transport in a device since separating degenerate valence bands with different curvatures allows for engineering a semiconductor to have a more preferrable hole effective mass. If we can move the heavy-hole band away from the light-hole band, then we can increase the number of light-holes in the device, which would increase mobility and increase current.

\begin{figure}[!ht] 
    \centering
    \includegraphics*[width = 6cm]{si-bands-thirdset.png}
    \caption{Silicon bandstructure, bulk, with spin-orbit coupling, with strain}
    \label{fig:si-third-set}
\end{figure}    

\subsection{Silicon Nanowire}

% TODO: consider a silicon nanowire and [100] transport direction.
% [] silicon nanowire [100]
%       [] vary diameter from 1nm to 7nm (step of 1nm)
%       [] extract bandgap and electron effective mass
%       [] plot versus diameter, comment on findings

Now, what if we consider Silicon nanowire? The \textit{Band Structure Lab} simulator allows us to calculate the bandstructures for such devices, and we can vary the diameter of the nanowire, and the direction in which the nanowire is built. So, we can vary diameter from 1nm-7nm for (100) and (110) direction nanowires.

Using Microsoft EXCEL, we can import the \(E-k\) data from \textit{nanoHUB}, and through some post-processing find \(E_g\) and \(m_n^*\) as functions of the nanowire diameter. Finding \(E_g\) is easy, we just subtract the CBM by the VBM. Finding \(m_n^*\) is requires more work.

Converting the \(E-k\) data to SI-units, we can plot the conduction band. From here, EXCEL makes a \(2^{nd}\)-degree polynomial trendline, which using Eq. \ref{eq:curvature-effective-mass} we can use calculate for the effective mass.

From Figure \ref{fig:bandgap-si-100}, we see that as the nanowire diameter increases, the bandgap energy \(E_g\) approaches 1.12eV.

\begin{figure}[!ht] 
    \centering
    \includegraphics*[width = 8cm]{bandgap-si-100.png}
    \caption{Bandgap versus Nanowire Diameter - Silicon (100)}
    \label{fig:bandgap-si-100}
\end{figure}    

From Figure \ref{fig:effmass-si-100}, as the nanowire diameter increases, the electron effective mass \(m_n^*\) increases. 

\begin{figure}[!ht] 
    \centering
    \includegraphics*[width = 8cm]{effmass-si-100.png}
    \caption{Electron Effective Mass versus Nanowire Diameter, Silicon (100)}
    \label{fig:effmass-si-100}
\end{figure}    

% TODO: consider silicon nanowire [110], and repeat above steps
%       [] Which nanowire has smallest effective mass?
%       [] How does this mass compare to the bulk counterpart?

From Figure \ref{fig:bandgap-si-110}, we can also see that as the nanowire diameter increases, the bandgap energy \(E_g\) approaches 1.12eV.

\begin{figure}[!ht] 
    \centering
    \includegraphics*[width = 8cm]{bandgap-si-110.png}
    \caption{Bandgap versus Nanowire Diameter - Silicon (110)}
    \label{fig:bandgap-si-110}
\end{figure}    

From Figure \ref{fig:effmass-si-110}, as the nanowire diameter increases, the electron effective mass \(m_n^*\) increases--quite drastically. From the bandstructures, the band around the CBM flattened greatly as nanowire diameter increased in the (110)-direction. 

\begin{figure}[!ht] 
    \centering
    \includegraphics*[width = 8cm]{effmass-si-110.png}
    \caption{Electron Effective Mass versus Nanowire Diameter, Silicon (110)}
    \label{fig:effmass-si-110}
\end{figure}    

It is reasonable to see \(E_g\) decrease, or relax, as we increase nanowire diameter. Silicon electrons in small nanowires experience large amounts of quantum confinement, and this confinement increases for smaller and smaller nanowires. Given a confinement width \(L\), we know from the Schr\"{o}dinger Equation, that \(k_n \propto 1/L\) and \(E_n \propto k^2 \), so \(E_n \propto 1 / L^2\), where \(E_n\) is a quantized energy state. This relationship follows as our nanowire diameter increases, the spatial confinement the electron feels increases, so energy decreases.

To explain the increase in electron effective mass, observe the density of states function in one-dimension:

\begin{equation}\label{eq:dos-1D}
    g(E)_{1D} = \frac{1}{\pi\hbar} \sqrt{\frac{m^*}{2E}}
\end{equation}

Increasing the size of the nanowire leads to more states for electrons to fill, but the states themselves do not change in concentration. However, the bandgap energy does decrease when the diameter increases. This means our electrons need less energy \(E\), and to compensate to keep \(g(E)\) the same, our effective mass must increase. This increase happens drastically in the (110)-direction, which relates to the K-point in the Brillouin zone--a place electrons are unlikely to be in Silicon. The increase in the (100)-direction settles around \(m_n^* \approx 1.6m_0\); the (100)-direction corresponds to the X-point, where the CBM naturally occurs for Silicon.

\section{Conclusion}

This lab is a perfect crash-course in learning how to interpret semiconductor properties from their bandstructure diagrams. I learned how to identify direct and indirect bandgap semiconductors, identifying negative electron effective mass, determining the shape of a semiconductor\'s constant energy surfaces, identifying and engineering degeneracy in the energy bands, determining hole properties in the valence band, and how all of this changes going from bulk semiconductors to nanowire semiconductors.

% REFERENCES!
\bibliographystyle{IEEEtran}
\bibliography{project2Bib.bib}

\end{document}
