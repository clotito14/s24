% Chase Lotito - ECE426 HW 5 - Latex

\documentclass{article}

\title{Relating oxygen vacancies to \(\beta-Ga_2 O_3\) unintentional n-type doping}
\author{Chase Lotito - SIUC UNDERGRADUATE}
\date{}

%% PACKAGES %%

\usepackage{amsmath, amsfonts, amssymb, amsthm}
\usepackage{braket}
\usepackage{listings}
\usepackage{geometry}
\usepackage{xcolor}
\usepackage{textcomp}
\usepackage{graphicx}
\usepackage{fancyhdr}
\usepackage{sourcecodepro}
\usepackage{multirow}

%%%%%%%%%%%%%%

\graphicspath{{./images}}
\setlength\parindent{0pt}       % globally supress indentation

%% LISTINGS CONFIG %%

\definecolor{purple2}{RGB}{153,0,153} % there's actually no standard purple
\definecolor{green2}{RGB}{0,153,0} % a darker green

\lstset{
  basicstyle=\normalsize\ttfamily,   % size of the fonts for the code
  frame = single,
  % Color settings to match IDLE style
  keywordstyle=\color{orange},       % core keywords
  keywordstyle={[2]\color{purple2}}, % built-ins
  stringstyle=\color{green2},%
  showstringspaces=false,
  commentstyle=\color{red},%
  upquote=true,                      % requires textcomp
  numbers=left,
  breaklines=true,
}

\begin{document}

%%%%%%%%%%%%%%%%%%%%%
\pagestyle{fancy}

\maketitle

% attempt to make nice header
\fancyhead{}
\fancyhead[CH]{\normalsize{SIUC / Chase Lotito / Spring 2024}}

Reading from \textit{"Recent progress on the electronic structure, defect, and doping properties of \(Ga_2 O_3\)"} by Jiaye Zhang and others, they have a section on unintentional n-type doping in \(\beta - Ga_2 O_3\). They then describe how there is no consensus on why this occurs, but give some merit to oxygen vacancies (\(V_o\)).

\subsection{Oxygen Vacancies}

Oxygen vacancies in metal-oxide semiconductors would occur during the crystal-growth process. In Floating Zone (FZ) growth, the growth environment would have some level of oxygen that can be controlled. Experimental data in the paper plotted the grown \(\beta-Ga_2 O_3\) crsytal's oconductivity to be inversely proportional to the oxygen flow rate in the growth environment, which could be related to the partial pressure of oxygen in the growth environment (\(P_{O_2}\)). 

\begin{equation}
        \sigma \propto \frac{1}{P_{O_2}}
\end{equation}

So the question is, why does this cause, or how can we show mathematically why oxygen vacancies (\(V_o\)) cause n-type conductivity. 

\bigskip

But, it can be explained with the Fermi Level \(E_F\).

\subsection{The Fermi Level}

In the textbook, we can describe the distance between the Fermi Level and the energy in the conduction band with this equation:

\begin{equation}
    E_c - E_F = kT \ln{\frac{N_c}{n_0}}
\end{equation}

Where we know that as the Fermi Level approaches the conduction band, as in \(E_c - E_F \to 0\), the material becomes n-type conductive.

\smallskip

Assuming that temperature is stable, we can see what happens to the effective density of states function in the conduction band \(N_c\). This parameter describing what energy states the electrons (our material's majority carrier) have to work with within the conduction band. 

\subsubsection{Connecting \(N_c\) to \(V_o\)}

Empirically, we see larger conductivities for lower oxygen rates, or more oxygen vacancies.

\begin{equation}
    \uparrow V_o \implies \uparrow \sigma
\end{equation}

Then we know that increases in electric conductivity is directly proportional to electron mobility.

\begin{equation}
    \sigma \propto \mu_n
\end{equation}

Electron mobility is given by this equation in the textbook:

\begin{equation}
    \mu_n = \frac{q \tau}{m^*_n}
\end{equation}

And given this equation for the effective density of states function in the conduction band from the textbook:

\begin{equation}
    N_c = 2 \left( \frac{2 \pi m^*_n k T}{h^2}\right)^{3/2}
\end{equation}

Which shows \(N_c \propto (m^*_n)^{3/2}\). So now we can make this connection.

\begin{equation}
    \uparrow V_o \implies \uparrow \sigma \implies \uparrow \mu_n \implies \downarrow m^*_n \implies \downarrow N_c \\
\end{equation}

Or:

\begin{equation} 
    V_o \propto 1 / N_c
\end{equation}

\subsubsection{Evaluating \(N_c \to n_0\)}

From experimental data \(N_c \sim 10^{19}\) and \(n_0 \sim 10^{17}\) for \(\beta-Ga_2 O_3\); if not, \(N_c > n_0\), generally.

\smallskip

If we assume this we can perform a limit on the distance between the conduction band and the Fermi level as \(N_c \to n_0\), which mimics the relationship of \(V_o\) and \(N_c\).

\begin{equation} 
    \lim_{N_c \to n_0} \left( E_c - E_F \right) \\ 
    = \lim_{N_c \to n_0} \left( kT \ln \frac{N_c}{n_0} \right) \\
    = k T \ln 1 \\
    = 0 \\
\end{equation}

So, the oxygen vanancies \(V_o\) cause the Fermi Level \(E_F\) to approach the conduction band energy \(E_c\), which is the property of a n-type material.

\end{document}
